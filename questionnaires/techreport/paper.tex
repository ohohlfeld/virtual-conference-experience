\documentclass[sigconf]{acmart}

\setcopyright{none}
\settopmatter{printacmref=false}
\renewcommand\footnotetextcopyrightpermission[1]{}

\pagestyle{plain}

\usepackage{paralist}
\usepackage{enumitem}%
\usepackage[english]{babel}
\usepackage[utf8]{inputenc}
\usepackage{subcaption}
\usepackage{wrapfig}
\usepackage{wasysym}%
\usepackage{forloop}%
\usepackage{xspace}

\newcommand{\afblock}[1]{\noindent{\textbf{#1 }}}
\newcommand{\takeaway}[1]{\noindent{\textit{\textbf{Takeaway:}}} \textit{#1}}

\newcommand{\confname}[0]{\texttt{SomeConf}\xspace}
\newcommand{\confloc}[0]{\texttt{SomeLocation}\xspace}

\usepackage[nolist]{acronym}

\begin{acronym}
\acro{NPS}{Net Promoter Score}
\acro{QoE}{Quality of Experience}
\end{acronym}

%
%
%
\newcommand{\QO}{$\Box$}%

%
%
%
\newenvironment{Qlist}{%
\renewcommand{\labelitemi}{\QO}
\begin{itemize}[leftmargin=1.5em,topsep=-.5em]
}{%
\end{itemize}
}

%
%
\newcommand{\Qline}[1]{\noindent\rule{#1}{0.6pt}}

%
%
\newcounter{ql}
\newcommand{\Qlines}[1]{\forloop{ql}{0}{\value{ql}<#1}{\vskip0em\Qline{\linewidth}}}


 

\title{A Questionnaire to Assess Virtual Conference Participation Experience}

\author{Oliver Hohlfeld}
\email{hohlfeld@b-tu.de}
\affiliation{\institution{Brandenburg University of Technology}}

\author{Dennis Guse}
\email{dennis.guse@alumni.tu-berlin.de}

\begin{teaserfigure}
	\parbox{\textwidth}{\centering\normalsize
		This articel is a technical report documenting the questionnaire not a scientific paper and has NOT been peer reviewed.\\
	}
	\vspace{10pt}
\end{teaserfigure}
\keywords{User experience, COVID-19, academic conferences}

\begin{abstract}
This technical report describes the preliminary design of a questionnaire to assess the participants experience of virtual conferences, compared to classical on-site conferences.
The survey approach consists of a pre-conference questionnaire assessing participation goals and a post-conference questionnaire assessing the participation experience.
It enables a data-driven investigation of the participant's experience, goals, and general feedback about virtual conferences.
Beyond its original goal of assessing virtual conferences, the questionnaire enables to assess goals of conference participation.
The questionnaire was successfully used at one conference and two workshop.
A HTML version and data is provided by a companion repository.
It is released early to foster research on virtual conferences.
\end{abstract}

\begin{document}
\maketitle

\sloppypar
\pagestyle{plain}

\section{Introduction}
The outbreak of the COVID-19 pandemic in 2020 forced academic conference to move to an online format with only little preparation time and prior experience.
This necessity for virtual conferences is rooted travel restrictions and social distancing enforced by many governments around the world as a response to the outbreak.

%
%
The dominant publication model of many communities (e.g., in computer science) relies on in-person conferences and workshops to disseminate work and to open-up new directions for future work.
Besides discussions also social bonds emerge that strengthen the scientific community.
In light of the measures to fight the pandemic, such in-person gathering were no longer possible and had to either canceled or moved to an online format (see e.g.,~\cite{bonifati2020holding, ASPLOS,alex2020flexibility}).
Once successfully established, virtual conferences might be the future~\cite{virtConferencesFuture}.
%
%
%
%
%
%
%
%
Up to the outbreak, only few venues experimented with means for remote attendance (e.g., using telepresence robots~\cite{RobotAttendance, RobotAttendance2018}).
The Internet Engineering Task Force (IETF) enables remote attendance for their meetings and few other conferences provide live streams (e.g., network operator meetings such as RIPE or the Chaos Communication Congress).
Despite these isolated efforts, the academic community has not largely moved to organizing all-virtual conferences---a new format~\cite{Price457} that needed to be realized from one day to the next as a result of the pandemic.

While the organization of on-site conferences and meetings follows established practices, largely moving academic conferences to an online format is new and thus little to no prior experience exists.
As a result, an ACM Presidential Task Force was formed in March to provide quick advice to conference organizers suddenly facing the need to move their conference online in light of the social distancing recommendations and global restrictions on travel due to the COVID-19 pandemic~\cite{virtConfTaskforce}.
While different design guidelines for organizing virtual conferences emerged, the participants experience is not yet understood. 
Do virtual conferences provide value beyond on-site conferences and might stay?
Do participants enjoy participating in virtual conferences?
Which aspects of current virtual conference designs do not yet work well and need improvement?

To shed light on these questions, we present a questionnaire to assess the participants experience of attending virtual conferences.
This enables a data-driven investigation of the participant's experience, goals, and general feedback about virtual conferences.
It assesses virtual conference attendance relative to prior experience in attending on-site conferences---and thus assumes that such experience exists (asked as part of the questionnaire).
This way, potential limitations, opportunities, and challenges of virtual conferences can be understood.
Lessons learned by surveying participants can be used to improve the design of future virtual conferences.
In the absence of established questionnaires to assess the participants experience of virtual conferences and to compare it to on-site versions, we designed the questionnaire in early March 2020 and have successfully applied it at one conferences~\cite{PAM2020} and two workshops~\cite{ItSecWorkshop, KuvsNetsoft}. 
We maintain a public repository~\cite{VirtualConferencesGithub} that contains raw data, analysis scripts, and an HTML version of this questionnaire implemented using TheFragebogen~\cite{TheFragebogen}.
While there are many angles for improvement and the reliability of the questionnaire cannot be assessed with only few applications yet, we already found it a useful tool to shed light on the participants experience of the three venues we used it so far.
We thus decided to share its preliminary design publicly to be used and extended.
Beyond its original goal of assessing virtual conferences, the questionnaire enables to assess goals of conference participation---aspects that haven't been analyzed in many communities and can likewise inform the design of on-site conferences.
With this, we aim to provide a tool for assessing the participants experience of academic conferences to the community and thereby aim at simulating research on virtual conferences.

\section{Questionnaire Design}
We propose two questionnaires to investigate the participants expectations and experiences.
The first questionnaire is called \textit{pre-conference questionnaire} and should be distributed before the virtual conference takes place.
The second questionnaire is called \textit{post-conference questionnaire} and should be administered directly after the virtual conference.

\afblock{Pre-Conference Questionnaire.}
The aim of this first questionnaire is to assess participant demographics, timezone, and primarily to gather data on goals of participation, the planned participation, and prior experience with virtual conferences.
It needs to be sent to all participants prior to the conference start.
We assume that the virtual conference is a replacement for an on-site conference and we want also to understand the expectations of participants who attended prior conferences.

The pre-conference questionnaire consisted of four sections.
First, we collected the participant's time zone, the professional seniority, and if participants already attended prior versions of this conference.
Subsequently, we assess the participants goals regarding their participation either as on-site conference and also as  virtual conference.
We assess the following goals: presenting their own work, following other researchers work, and also social interaction.
Social interaction is split into meeting already known researchers and meeting new researchers.
Also, we assess how participants plan to their conference participation, e.g., which tools they plan to use and which sessions they plan to attend.
These information can inform conference organizers before the start of the venue.
Since our goal is to compare the virtual conference experience to on-site venues, we asked the participants to provide the same information but in the context of attending on-site conferences.
This way, the questionnaire is a self-report study.
We remark that self-report studies are an often used tool (e.g., in health and psychology research), yet the accuracy and reliability of reports of one's own behavior is unknown and thus self-report studies can have validity problems.

The pre-conference questionnaire is shown in Appendix~\ref{sec:preconf}.

\afblock{Post-Conference Questionnaire.}
The post-conference questionnaire assesses the actual experience---instead of the planned behavior and goals---after conference attendance.
It thus needs to be answered by participants after the conference ended, e.g., sent by the general chairs right after the closing ceremony.
It assesses the participants' experience of attending the conference by collecting the following data:
\begin{inparaenum}[\it i)]
    \item Participant information (same as pre-conference questionnaire).
    \item Overall experience of attending presentation sessions.
    \item Presentation-related interactions.
    \item Social interactions (Virtual Hallway Track).
    \item Overall experience and fulfilment of expectations.
    \item General comments about virtual attendance and suggestions for future virtual conferences.
\end{inparaenum}

We assume that both questionnaires are answered anonymously to increase the response rate.
We thus assess the same participant information in both the pre- and the post-conference questionnaire.
An alternative implementation of our approach can assign pseudonyms (e.g., tokens) to participants that enable both questionnaires to be correlated.
We opted for a simpler design and did not use pseudonyms since the correlation of both questionnaires was not subject to our intended analysis.
Yet, the responses to both the participant information parts in both questionnaires can be compared statistically to check if the populations that answered the questionnaires are statistically similar.

The overall experience assessed overall quality on a continuous 7-point scale defined in~\cite{itu-t_recommendation_p.851_subjective_2003}---an approach that is often used in \ac{QoE} research.
This scale is an extension of a discrete ACR 5-point scale that is recommended by the ITU-T in Rec.\ P.800 (bad to excellent) in which the two extrema need to be displayed differently since they are supposed not be directly selected (see the implementation in TheFragebogen~\cite{TheFragebogen}).
If such a display is not available, the overall quality should be assessed using the 5-point scale.
This section of the questionnaire further assesses to which degree the expectations described in the pre-conference questionnaire and those on the virtual conference were met.
Therefore, the fulfillment of the participants expectations is collected on the same dimensions as in the pre-conference questionnaire.
It also asks if one would want to attend more conferences if they are offered online.

The next section assesses the attendance of talk sessions.
It assesses which tools were used and how well they worked and how the talk sessions compare to on-site versions of the conference (or other conferences).
The latter dimensions are assessed on a 5-point Likert scale (much worse to much better).
We assume that participants have attended at least one on-site conference prior to the assessed virtual conference (thus prior experience is assessed in the participant information). 
In includes optional textual input on the pros and cons of the use of pre-recorded talks---a format used by virtual conferences to mitigate technical problems during live streams.
Last, it asked how many sessions were skipped due to time zone difference---a main challenge that virtual conference face when providing live streams to a global audience.

We assess the interaction in two sections: \emph{i)} presentation-related interactions and \emph{ii)} social Interactions (Virtual Hallway Track).
The interaction quality is assessed on a 5-point Likert scale (much worse to much better)---the same as in the previous session to keep the overall number of scales minimal.
It is again asked relative to classical on-site conferences to assess the benefit / drawback of virtual conferences.
Again, both sections contain optional textual input fields to assess the respective positive and negative aspects of interactions and ways for improvement.

%
%
%

The last section (general comments about remote attendance) asked for general feedback (textual input) and the experience with the virtual conference format.
The virtual conference format is assessed using the following scales.
First, the agreement to the statement to attend more virtual conferences in the future.
Second, the share of virtual conferences in the future (from 0\%---all in-person---to 100\%---all virtual).
Third, the \ac{NPS} to investigate the impact on customer loyalty \cite{reichheld_one_2003}.
The \ac{NPS} asks how likely it would be that attending this virtual conference would be recommended to colleagues (0 \emph{not likely at all} until 10 \emph{extremely likely}).
While it is of questionable reliability~\cite{npscritism} and not well-established the \ac{QoE} domain, it is popular in marketing and user retention analyses.

The post-conference questionnaire is shown in Appendix~\ref{sec:postconf}.
We provide a HTML version implemented using TheFragebogen~\cite{TheFragebogen} at \cite{VirtualConferencesGithub} to ease its direct application to further conferences.
Parts of the questionnaire (e.g., on the used tools) need to be adapted to the assessed conference before use.

\balance

\clearpage
\bibliographystyle{acm}
\bibliography{references}

\begin{appendix}
\clearpage
%
\section{Pre-Conference Questionnaire}
\label{sec:preconf}

\newcounter{preConfCounter}

Remote conference attendance is a new experience for all of us and this is the very first time that \confname is held as a virtual conference. To understand your expectations and to optimize virtual conferences in the future, we would like to ask you to complete this questionnaire as part of research on virtual conference attendance. It is complemented by a post-conference questionnaire to assess your conference experience that we will sent after the conference. We appreciate and value your effort in helping with this research. The goal is to derive community guidelines for virtual conferences. With your answers, you will help to shape future conferences!

This questionnaire consists of 5 blocks and will take you about 10 minutes to complete:
1) Personal details
2) Intended participation
3) Your goals when attending a on-site conference (e.g., prior editions of \confname)
4) Your goals when attending \confname as virtual conference
5) Other comments (optional)

All answers are collected anonymously. We thank you for your input!

\subsection{Personal details}
We first would like to learn few personal details about you to be able to better interpret your answers in this questionnaire.

\begin{enumerate}

	\item How many \confname conferences have you attended, including this one?
		\begin{Qlist}
			\item 1
			\item 2
			\item 3
			\item 4--6
			\item $>6$
		\end{Qlist}

	\item In which timezone are you?
		\begin{Qlist}
			\item (GMT -12:00) Eniwetok, Kwajalein
			\item (GMT -11:00) Midway Island, Samoa
			\item (GMT -10:00) Hawaii
			\item (GMT -9:00) Alaska
			\item (GMT -8:00) Pacific Time (US \& Canada)
			\item (GMT -7:00) Mountain Time (US \& Canada)
			\item (GMT -6:00) Central Time (US \& Canada), Mexico City
			\item (GMT -5:00) Eastern Time (US \& Canada), Bogota, Lima
			\item (GMT -4:00) Atlantic Time (Canada), Caracas, La Paz
			\item (GMT -3:30) Newfoundland
			\item (GMT -3:00) Brazil, Buenos Aires, Georgetown
			\item (GMT -2:00) Mid-Atlantic
			\item (GMT -1:00) Azores, Cape Verde Islands
			\item (GMT) Western Europe Time, London, Lisbon, Casablanca
			\item (GMT +1:00) Brussels, Copenhagen, Madrid, Paris
			\item (GMT +2:00) Kaliningrad, South Africa
			\item (GMT +3:00) Baghdad, Riyadh, Moscow, St. Petersburg
			\item (GMT +3:30) Tehran
			\item (GMT +4:00) Abu Dhabi, Muscat, Baku, Tbilisi
			\item (GMT +4:30) Kabul
			\item (GMT +5:00) Ekaterinburg, Islamabad, Karachi, Tashkent
			\item (GMT +5:30) Bombay, Calcutta, Madras, New Delhi
			\item (GMT +6:00) Almaty, Dhaka, Colombo
			\item (GMT +7:00) Bangkok, Hanoi, Jakarta
			\item (GMT +8:00) Beijing, Perth, Singapore, Hong Kong
			\item (GMT +9:00) Tokyo, Seoul, Osaka, Sapporo, Yakutsk
			\item (GMT +9:30) Adelaide, Darwin
			\item (GMT +10:00) Eastern Australia, Guam, Vladivostok
			\item (GMT +11:00) Magadan, Solomon Islands, New Caledonia
			\item (GMT +12:00) Auckland, Wellington, Fiji, Kamchatka
		\end{Qlist}

	\item From where are you participating in \confname?
		\begin{Qlist}
			\item Office / workplace
			\item Home
			\item Commuting
			\item Co-working space
			\item Other \Qline{3cm}
		\end{Qlist}

	\item Please rate your professional seniority
		\begin{Qlist}
			\item Undergrad / Master student
			\item PhD student
			\item Postdoc
			\item Faculty
			\item Industry
			\item Other \Qline{3cm}
		\end{Qlist}

	\item Have you attended a remote conference before?
		\begin{Qlist}
			\item No, this is my first remote conference
			\item Yes
		\end{Qlist}

	\item Would you have attended \confname as on-site conference in \confloc?
		\begin{Qlist}
			\item Yes, I would have physically attended \confname in \confloc
			\item No, I am only attending because of the remote participation and would not have come to \confloc
		\end{Qlist}

\setcounter{preConfCounter}{\value{enumi}}
\end{enumerate}

\subsection{Planned participation}
How do you plan to participate in the \confname virtual conference?

\begin{enumerate}
\setcounter{enumi}{\value{preConfCounter}}
	\item Do you intend to use the conference Slack/Zoom as virtual hallway track?
		\begin{Qlist}
			\item Yes
			\item No
			\item I have not decided yet
		\end{Qlist}

	\item In which sessions do you plan to participate?
		\begin{Qlist}
			\item Session 1 (replace with real name)
			\item Session 2 (replace with real name)
			\item $\cdots$
		\end{Qlist}

	\item Do you think you would attend more sessions if you would have attended \confname as on-site venue?
		\begin{Qlist}
			\item Yes, I would have attended more
			\item No
		\end{Qlist}
\setcounter{preConfCounter}{\value{enumi}}
\end{enumerate}

\clearpage
\subsection{Your goals when attending a on-site conference (e.g., prior editions of \confname)}
To begin with, please tell us about your goals when attending an on-site conference in person (such as \confname in the previous years). We will ask you to your goals when attending \confname as online conference in the next section.

\begin{enumerate}
\setcounter{enumi}{\value{preConfCounter}}
	\item Importance of presenting my work to this community
		\begin{tabular}{p{2cm}p{.8cm}p{.9cm}p{.9cm}p{.9cm}p{.9cm}}
			& Not at all & Slightly & Moderate & Very & Extremely\\
			 
			Importance of presenting my work to this community & \QO & \QO & \QO & \QO & \QO \\
\hline
			Importance of following paper/poster presentations & \QO & \QO & \QO & \QO & \QO \\
\hline
			Importance of interacting with other researchers I already know & \QO & \QO & \QO & \QO & \QO \\
\hline
			Importance of meeting new researchers & \QO & \QO & \QO & \QO & \QO \\
\hline
			Importance of visiting the city/country in which the conference is held & \QO & \QO & \QO & \QO & \QO \\
		\end{tabular}
\setcounter{preConfCounter}{\value{enumi}}
\end{enumerate}

\subsection{Your goals when attending \confname as virtual conference}
What are you goals when attending this virtual edition of \confname?

\begin{enumerate}
\setcounter{enumi}{\value{preConfCounter}}
	\item Importance of presenting my work to this community
		\begin{tabular}{p{2cm}p{.8cm}p{.9cm}p{.9cm}p{.9cm}p{.9cm}}
			& Not at all & Slightly & Moderate & Very & Extremely\\
			 
			Importance of presenting my work to this community & \QO & \QO & \QO & \QO & \QO \\
\hline
			Importance of following paper/poster presentations & \QO & \QO & \QO & \QO & \QO \\
\hline
			Importance of interacting with other researchers I already know & \QO & \QO & \QO & \QO & \QO \\
\hline
			Importance of meeting new researchers & \QO & \QO & \QO & \QO & \QO \\
		\end{tabular}
\setcounter{preConfCounter}{\value{enumi}}
\end{enumerate}


\subsection{Other comments (optional)}
\begin{enumerate}
\setcounter{enumi}{\value{preConfCounter}}
	\item Do you have other feedback / wishes / comments that you would like to share?
		\Qlines{3}
\end{enumerate}
 \clearpage
%
\section{Post-Conference Questionnaire}
\label{sec:postconf}

\newcounter{postConfCounter}

Thank you for participating in \confname! Since this was the very first time that \confname was held as virtual conference, we are interested in how you experienced \confname in the broader context of research on virtual conferences. We therefore kindly ask you to participate in this post- conference questionnaire.

This questionnaire consists of 6 blocks and will take you about 15-20 minutes to complete: 1) Personal details (as in the pre-conf questionnaire since we cannot correlate both due to anonymity)
2) Overall Experience
3) Talk Sessions
4) Presentation-related Interactions
5) Social Interactions (Virtual Hallway Track)
6) General comments about virtual attendance

All answers are collected anonymously and we will share the results with you afterwards. We thank you for your input.

\subsection{Personal details}
We first would like to learn few personal details about you to be able to better interpret your answers in this questionnaire.

\begin{enumerate}

	\item How many \confname conferences have you attended, including this one?
		\begin{Qlist}
			\item 1
			\item 2
			\item 3
			\item 4--6
			\item $>6$
		\end{Qlist}

	\item In which timezone are you?
		\begin{Qlist}
			\item (GMT -12:00) Eniwetok, Kwajalein
			\item (GMT -11:00) Midway Island, Samoa
			\item (GMT -10:00) Hawaii
			\item (GMT -9:00) Alaska
			\item (GMT -8:00) Pacific Time (US \& Canada)
			\item (GMT -7:00) Mountain Time (US \& Canada)
			\item (GMT -6:00) Central Time (US \& Canada), Mexico City
			\item (GMT -5:00) Eastern Time (US \& Canada), Bogota, Lima
			\item (GMT -4:00) Atlantic Time (Canada), Caracas, La Paz
			\item (GMT -3:30) Newfoundland
			\item (GMT -3:00) Brazil, Buenos Aires, Georgetown
			\item (GMT -2:00) Mid-Atlantic
			\item (GMT -1:00) Azores, Cape Verde Islands
			\item (GMT) Western Europe Time, London, Lisbon, Casablanca
			\item (GMT +1:00) Brussels, Copenhagen, Madrid, Paris
			\item (GMT +2:00) Kaliningrad, South Africa
			\item (GMT +3:00) Baghdad, Riyadh, Moscow, St. Petersburg
			\item (GMT +3:30) Tehran
			\item (GMT +4:00) Abu Dhabi, Muscat, Baku, Tbilisi
			\item (GMT +4:30) Kabul
			\item (GMT +5:00) Ekaterinburg, Islamabad, Karachi, Tashkent
			\item (GMT +5:30) Bombay, Calcutta, Madras, New Delhi
			\item (GMT +6:00) Almaty, Dhaka, Colombo
			\item (GMT +7:00) Bangkok, Hanoi, Jakarta
			\item (GMT +8:00) Beijing, Perth, Singapore, Hong Kong
			\item (GMT +9:00) Tokyo, Seoul, Osaka, Sapporo, Yakutsk
			\item (GMT +9:30) Adelaide, Darwin
			\item (GMT +10:00) Eastern Australia, Guam, Vladivostok
			\item (GMT +11:00) Magadan, Solomon Islands, New Caledonia
			\item (GMT +12:00) Auckland, Wellington, Fiji, Kamchatka
		\end{Qlist}

	\item From where are you participating in \confname?
		\begin{Qlist}
			\item Office / workplace
			\item Home
			\item Commuting
			\item Co-working space
			\item Other \Qline{3cm}
		\end{Qlist}

	\item Please rate your professional seniority
		\begin{Qlist}
			\item Undergrad / Master student
			\item PhD student
			\item Postdoc
			\item Faculty
			\item Industry
			\item Other \Qline{3cm}
		\end{Qlist}

	\item Have you attended a remote conference before?
		\begin{Qlist}
			\item No, this is my first remote conference
			\item Yes
		\end{Qlist}

	\item Would you have attended \confname as on-site conference in \confloc?
		\begin{Qlist}
			\item Yes, I would have physically attended \confname in \confloc
			\item No, I am only attending because of the remote participation and would not have come to Oregon
		\end{Qlist}

\setcounter{postConfCounter}{\value{enumi}}
\end{enumerate}


\subsection{Overall Experience}
This block is about assessing your overall experience of attending \confname as virtual conference. Did you enjoy it?

\begin{enumerate}
\setcounter{enumi}{\value{postConfCounter}}

	\item My overall experience of attending remotely \confname is
		\begin{Qlist}
			\item extremely bad
			\item bad
			\item poor
			\item fair
			\item good
			\item excellent
			\item ideal
		\end{Qlist}

	\item Compared to attending on-site, attending remotely allowed me to better focus on the conference

		\begin{tabular}{p{2cm}p{.8cm}p{.9cm}p{.9cm}p{.9cm}p{.9cm}}
			\hline
			& Strongly Disagree & Disagree & Undecided & Agree & Strongly Agree\\
			 
			I was more focused on the technical content & \QO & \QO & \QO & \QO & \QO \\
		\end{tabular}


	\item My expectations on attending a virtual conference were met 

		\begin{tabular}{p{2cm}p{1.5cm}p{1.5cm}p{1.5cm}}
			\hline
			& Better than expected & As expected & Less than expected\\
			The possibility to \emph{follow presentations} was & \QO & \QO & \QO\\
			\hline
			The possibility to \emph{interact with speakers} was & \QO & \QO & \QO\\
			\hline
			The possibility to \emph{interact with participants} was  & \QO & \QO & \QO\\
		\end{tabular}

	\item (Optional) To speakers: My expectations on attending a virtual conference were met

		\begin{tabular}{p{2cm}p{1.5cm}p{1.5cm}p{1.5cm}}
			\hline
			& Better than expected & As expected & Less than expected\\
			The possibility to \emph{present my work} was & \QO & \QO & \QO\\
		\end{tabular}

	\item Would you attend more conferences each year if they are offered online
		\begin{Qlist}
			\item Yes
			\item Maybe
			\item No
			\item Other \Qline{3cm}
		\end{Qlist}

\setcounter{postConfCounter}{\value{enumi}}
\end{enumerate}


\subsection{Talk Sessions}
In this section, we are interested how you experienced the talks sessions (i.e., paper and poster presentations).

\begin{enumerate}
\setcounter{enumi}{\value{postConfCounter}}

	\item What tools did you use to participate in the conference and how well did they work?
		\begin{tabular}{p{2cm}p{1cm}p{1cm}p{1cm}p{1cm}}
			\hline
			& Better than expected & As expected & Less than expected & Did not use\\
			Zoom & \QO & \QO & \QO & \QO\\
			\hline
			Slack & \QO & \QO & \QO & \QO\\
		\end{tabular}

	\item How do you rate the presentation sessions of this virtual conference compared to on-site conferences?
		\begin{tabular}{p{2cm}p{.8cm}p{.8cm}p{.8cm}p{.8cm}p{.8cm}}
			\hline
			& much worse & some\-what worse & stayed the same & some\-what better & much better\\
			The ability to attend talks was & \QO & \QO & \QO & \QO & \QO \\
			\hline
			The ability to focus on the talks without distractions was & \QO & \QO & \QO & \QO & \QO \\
			\hline
			The ability to ask questions was &  \QO & \QO & \QO & \QO & \QO \\
			\hline
			The ability to deeply discuss a topic openly with all participants was &  \QO & \QO & \QO & \QO & \QO \\
		\end{tabular}

	\item (optional) Pros: What did you like about pre-recorded talks?
		\Qlines{3}

	\item (optional) Cons: What did you not like about pre-recorded talks?
		\Qlines{3}

	\item How many paper sessions did you skip due to time zone differences
		\begin{Qlist}
			\item 0 (I followed every session)
			\item 1
			\item 2
			\item 3 (I skipped all sessions)
		\end{Qlist}

\setcounter{postConfCounter}{\value{enumi}}
\end{enumerate}


\subsection{Presentation-related Interactions}
This section is about how you interacted with other participants about the content presented at the conference in terms of papers and posters.

\begin{enumerate}
\setcounter{enumi}{\value{postConfCounter}}

	\item Did you participate in technical discussions on Slack or Zoom? 
		\begin{Qlist}
			\item Actively (I posted content to Slack or Zoom)
			\item Passively (I read what others posted)
			\item Other \Qline{3cm}
		\end{Qlist}

	\item How do you rate the interaction with others during this virtual conference compared to on-site conferences?
		\begin{tabular}{p{2cm}p{.8cm}p{.8cm}p{.8cm}p{.8cm}p{.8cm}}
			\hline
			& much worse & some\-what worse & stayed the same & some\-what better & much better\\
			The ability to ask questions was & \QO & \QO & \QO & \QO & \QO \\
			\hline
			The moderation of questions by the session chairs when posting to the Zoom chat was & \QO & \QO & \QO & \QO & \QO \\
			\hline
			The ability for detailed technical discussions was &  \QO & \QO & \QO & \QO & \QO \\
		\end{tabular}

	\item (optional): Pros: Other feedback on \emph{positive} aspects of the ability to interact during this virtual conference
		\Qlines{3}

	\item (optional): Cons: Other feedback on \emph{negative} aspects of the ability to interact during this virtual conference
		\Qlines{3}

\setcounter{postConfCounter}{\value{enumi}}
\end{enumerate}

\clearpage
\subsection{Social Interactions (Virtual Hallway Track)}
Besides content related to technical presentations (papers / posters), conferences are about general interactions with known and new researchers (e.g., during breaks or at the dinner table): the hallway track. We are interested in this section how you experienced the hallway track at \confname.

\begin{enumerate}
\setcounter{enumi}{\value{postConfCounter}}

	\item Did you use Slack or the Zoom chat to interact with others about topics unrelated to the presentations?
		\begin{Qlist}
			\item Yes
			\item No
		\end{Qlist}

	\item How do you rate the quality of social interactions (unrelated to the presentations) during this virtual conference compared to classical on-site conferences?

		\begin{tabular}{p{2cm}p{.8cm}p{.8cm}p{.8cm}p{.8cm}p{.8cm}}
			\hline
			& much worse & some\-what worse & stayed the same & some\-what better & much better\\
			The interactions with researchers I already know were & \QO & \QO & \QO & \QO & \QO \\
			\hline
			The interactions with new researchers I didn't know before were & \QO & \QO & \QO & \QO & \QO \\
		\end{tabular}

	\item (optional) What did you enjoy about social interaction at this virtual conference?
		\Qlines{3}

	\item (optional) How can social interactions be improved at virtual conferences?
		\Qlines{3}

\setcounter{postConfCounter}{\value{enumi}}
\end{enumerate}


\subsection{General comments about remote attendance}

\begin{enumerate}
\setcounter{enumi}{\value{postConfCounter}}

	\item I would like to attend more virtual conferences in the future
		\begin{Qlist}
			\item Strongly Disagree
			\item Disagree
			\item Undecided
			\item Agree
			\item Strongly Agree
		\end{Qlist}

	\item How many venues should be virtual in the future?
		\begin{Qlist}
			\item 0\% (all in person, please)
			\item 10\%
			\item 20\%
			\item 30\%
			\item 40\%
			\item 50\%
			\item 60\%
			\item 70\%
			\item 80\%
			\item 90\%
			\item 100\% (all virtual, please)
		\end{Qlist}

	\item How likely is it that you would recommend attending \confname as virtual conference to a colleague?
		\begin{Qlist}
			\item 0 (not all all likely to recommend)
			\item 1
			\item 2
			\item 3
			\item 4
			\item 5
			\item 6
			\item 7
			\item 8
			\item 9
			\item 10 (extremely likely to recommend)
		\end{Qlist}

	\item Pros: What worked well about attending remotely
		\Qlines{3}

	\item Cons: What did not work well about attending remotely
		\Qlines{3}

	\item Other comments / suggestions / feedback
		\Qlines{3}
\setcounter{postConfCounter}{\value{enumi}}
\end{enumerate}
 \end{appendix}


\end{document}
